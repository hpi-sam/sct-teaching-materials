%!TEX root = ../spezifikation.tex


\section*{Hinweise zum Entwurfsdokument}

Es handelt sich bei diesem Dokument um die \textbf{stark gekürzte} Fassung eines objektorientierten Entwurfs, wie in der Vorlesung vorgestellt und im vorangegangenen Projekt behandelt. 

Der Entwurf beginnt mit \autoref{sec:struktur} (Struktur der Komponenten).
Sämtliche für \docAssignmentTitle{} nicht direkt benötigten Informationen wurden herausgelassen.




\section*{Einleitung}

Es soll die Steuersoftware für Roboter in einer automatischen Paketsortieranlage entworfen werden.
Diese Steuersoftware soll dabei den Transport von Paketen von Abfertigungsstationen (kurz Stationen), an denen die Roboter beladen werden, zu verschiedenen Abwurfschächten gewährleisten. 
Ebenfalls zu berücksichtigen sind hier die Rückfahrten der Roboter zu den Stationen sowie Fahrten innerhalb der Stationen. 

Es werden über das Netzwerk Fahraufträge an die Roboter übermittelt. 
Die Aufträge beinhalten immer eine Zielkoordinate sowie die Art des Auftrags.
Die Steuersoftware der Roboter realisiert daraufhin autonom (d.h. nur auf Basis der lokalen Sensorinformationen) die Anfahrt zur gegebenen Koordinate.
Bei manchen Arten von Aufträgen muss zudem nach Ankunft noch eine Aktion durchgeführt werden (z.B. das Abladen eines Pakets).

Das Fahrverhalten der Roboter orientiert sich an der deutschen Straßenverkehrsordnung.


% Mögliche Aufträge sind
% \begin{enumerate*}[a)]
% 	\item das Anfahren eines Abwurfschacht zwecks Abwurf der Ladung
% 	\item die Rückfahrt zu einer Station
% 	\item die Fahrt innerzu einer Station
% \end{enumerate*}.


